
\section{Lý thuyết phạm trù}
\begin{define}
    Một \textdef{phạm trù} $\mathcal{C}$ được cấu thành từ ba thành phần:
    \begin{enumerate}[(i)]
        \item Họ các \textdef{vật} $\ob(\mathcal{C})$.
        \item Tập các \textdef{cấu xạ} $\Hom(A,B)$ ứng với mỗi cặp vật $(A,B)$.
        \item \textdef{Phép hợp nối} $\circ: \Hom(A,B) \times \Hom(B,C) \rightarrow \Hom(A,C)$.
    \end{enumerate}
    Trong đó ta kí hiệu một cấu xạ $f \in \Hom(A,B)$ là $f: A \rightarrow B$ hay $A \xrightarrow{f} B$. Hợp nối giữa hai cấu xạ $\circ(f,g)$ thường được kí hiệu là $g \circ f$ hay $gf$. Các thành phần này phải thỏa các tính chất:
    \begin{enumerate}[(i)]
        \item $\Hom(A,B) \cap \Hom(A',B') = \varnothing$, với mọi $(A,B) \neq (A',B')$. Nghĩa là mỗi cấu xạ $f \in \Hom(A,B)$ có duy nhất tập nguồn $A$ và tập đích $B$.
        \item Ứng với mỗi vật $A$, tồn tại \textdef{cấu xạ đơn vị} $1_A \in \Hom(A,A)$ thỏa $f 1_A = 1_A f = f$ với mọi $f: A \rightarrow B$.
        \item Phép hợp nối có tính kết hợp, nghĩa là với các cấu xạ $A \xrightarrow{f} B \xrightarrow{g} C$ thì
              $$
                  h(gf) = (hg)f.
              $$
    \end{enumerate}
    Nếu $\ob{\mathcal{C}}$ là tập hợp thì ta nói $\mathcal{C}$ là \textdef{phạm trù nhỏ}.
\end{define}

\begin{example}
    Phạm trù \textbf{Grp} là phạm trù mà các vật là các nhóm, cấu xạ là đồng cấu và phép hợp nối là phép hợp nối ánh xạ thông thường.

    Tương tự ta cũng có phạm trù $_R$\textbf{Mod} (tương ứng \textbf{Mod}$_R$) là phạm trù các $R$-môđun trái (tương ứng phải) với cấu xạ là đồng cấu $R$-môđun.

    Phạm trù \textbf{Top} là phạm trù mà các vật là không gian tôpô và cấu xạ là ánh xạ liên tục.
\end{example}

\begin{define}
    Cho $\mathcal{C}$ và $\mathcal{D}$ là hai phạm trù, khi đó một \textdef{hàm tử} $T: \mathcal{C} \rightarrow \mathcal{D}$ là một hàm biến vật thành vật, cấu xạ thành cấu xạ thỏa:
    \begin{enumerate}[(i)]
        \item Nếu $A \in \ob(\mathcal{C})$ thì $T(A) \in \ob(\mathcal{D})$
        \item Nếu $f: A \rightarrow B$ là cấu xạ trong $\mathcal{C}$ thì $T(f): T(A) \rightarrow T(B)$ là cấu xạ trong trong $\mathcal{D}$.
        \item Nếu $A \xrightarrow{f} B \xrightarrow{g} C$ là các cấu xạ trong $\mathcal{C}$ thì $T(A) \xrightarrow{T(f)} T(B) \xrightarrow{T(g)} T(c)$ là các cấu xạ trong $\mathcal{D}$ và
              $$
                  T(gf) = T(g)T(f)
              $$
        \item $T(1_A) = 1_{T(A)}$ với mọi $A \in \ob(\mathcal{C})$.
    \end{enumerate}
\end{define}

\begin{define}
    \textdef{Tập tiền thứ tự} $I$ là một tập được trang bị quan hệ $\leq$ có tính phản xạ và bắc cầu.

    Nếu $I \neq \varnothing$ và thỏa thêm tính chất: hai phần tử $x,y$ bất kì trong $I$ đều có chặn trên, nghĩa là tồn tại $z \in I$ sao cho $x \leq z$ và $y \leq z$ thì ta gọi $(I,\leq)$ là \textdef{tập định hướng}.
\end{define}

\begin{remark}
    Tập tiền thứ tự $(I,\leq)$ có thể được coi là một phạm trù theo nghĩa $\ob(I) = I$,
    $$
        \Hom(x,y) = \begin{cases}
            \varnothing    & \text{nếu } x \not\leq y \\
            \{\iota_{xy}\} & \text{nếu } x \leq y
        \end{cases}
    $$
    Phép hợp nối được định nghĩa bởi $\iota_{yz} \iota_{xy} = \iota_{xz}$, trong đó $x \leq y$ và $y \leq z$ với $x,y,z\in I$. Quan hệ phản xạ $x \leq x$ được coi là cấu xạ đơn vị trong phạm trù này ($\iota_{xx} \in \Hom(x,x)$).

    Ngược lại, một phạm trù nhỏ thỏa $|\Hom(x,y)| \leq 1$ và $|Hom(x,x)| = 1$ có thể coi là một tập tiền thứ tự với các phần tử là các vật và $x \leq y$ khi và chỉ khi $|\Hom(x,y)| = 1$.
\end{remark}

\begin{define}
    Cho $(I,\leq)$ là tập tiền thứ tự và $\mathcal{C}$ là phạm trù bất kì. Một \textdef{hệ thuận} trong $\mathcal{C}$ (trên $I$) là một họ các vật $\{X_i\}_{i \in I}$ và cấu xạ $\{ \phi_{ij}: X_i \rightarrow X_j \}_{i \leq j}$ sao cho $\phi_{ii} = 1_{X_i}$ và nếu $i \leq j \leq k$ thì biểu đồ sau giao hoán
    $$
        \begin{tikzcd}
            X_i \arrow[rr, "\phi_{ij}"] \arrow[rd, "\phi_{ik}"'] &     & X_j \arrow[ld, "\phi_{jk}"] \\
            & X_k &
        \end{tikzcd}
    $$
    Để ý ta có thể phát biểu ngắn gọn hơn rằng một hệ thuận trong $\mathcal{C}$ là một hàm tử $X: (I,\leq) \rightarrow \mathcal{C}$, trong đó ta kí hiệu $X_i = X(i)$ và $\phi_{ij} = X(\iota_{ij})$.
    % là một hàm tử $T: I \rightarrow \mathcal{C}$, trong đó ta kí hiệu $X_i = T(i)$ và $\phi_{ij} = T(\iota_{ij})$. Nói một cách cụ thể hơn thì một hệ thuận
\end{define}

\begin{define}\label{def:direct-limit}
    Cho $(I,\leq)$ là tập tiền thứ tự, $\mathcal{C}$ là phạm trù và $\{X_i, \phi_{ij}\}$ là hệ thuận trong $\mathcal{C}$ trên $I$. \textdef{Giới hạn thuận} của hệ trên là một vật $X = \varinjlim X_i$ và các cấu xạ $f_i: X_i \rightarrow X$ sao cho $f_j \circ \phi_{ij} = f_i$. Hơn nữa chúng phổ dụng với tính chất trên theo nghĩa, nếu $Y$ là một vật và $g_i: X_i \rightarrow Y$ là họ các cấu xạ thỏa $g_j \circ \phi_{ij} = g_i$ thì khi đó tồn tại duy nhất một cấu xạ $g: X \rightarrow Y$ sao cho $g_i = g \circ f_i$.

    Định nghĩa trên có thể được tóm tắt thông qua biểu đồ giao hoán sau

    $$
        \begin{tikzcd}
            X_i \arrow[rr, "\phi_{ij}"] \arrow[rd, "f_i"] \arrow[rdd, "g_i"', bend right] &                          & X_i \arrow[ld, "f_j"'] \arrow[ldd, "g_j", bend left] \\
            & X \arrow[d, "h", dashed] &                                                      \\
            & Y                        &
        \end{tikzcd}
    $$
\end{define}